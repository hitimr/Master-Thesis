\section{Discussion and Outlook}

As shown in Fig. \ref{fig:results} NanoVDB does provide better performance compared to OpenVDB on both platforms.
Since both frameworks use different data structures significant adaptations to the code base are required for a project to migrate to NanoVDB. 
The expected increase in performance for CPU based systems can be considered too small to justify the amount of necessary work.

However for machines with a dedicated GPU a switch to NanoVDB can be very attractive especially for large simulations.
Furthermore no adaptations to the codebase are required to launch kernels on either CPUs or GPUs.
This also allows to combine both platforms to perform calculations on both in parallel.

With its current implementation the main limiting factor of NanoVDB seems to be the kernel for ray intersections which is not well optimized GPUs as it includes 4 branches (2 if and 2 while).
Branches are usually no problem for CPUs due to their branch-prediction functionality.
However on GPUs with a SIMT \footnote{Single Instruction Multiple Thread} Architecture 1 branch may force multiple threads to idle until all branches in a thread block are merged again.

Since 2018 NVIDIA is also selling GPUs with hardware accelerated raytracing capabilities (RTX 20X0 and 30X0 series).
Due to its proprietary license the exact functionality of this technology is not part of the public domain.
But a recent analysis \cite{sanzharov2020survey} shows that a performance increase of up to one order of magnitude is possible in certain scenarios.
However additional research and testing is required to determine of NanoVDB is compatible with this technology.
