\section{Discussion and Outlook}
\subsection{Performance}
As shown in Fig. \ref{fig:results} NanoVDB does provide better performance compared to OpenVDB on both platforms.
While NVIDIA promises in increase of performance of approx 60x (see Tab. \ref{tab:nvidia_benchmark}), this benchmark only performed 6.2x faster.
This is likely due to NVIDIA trying to sell best-case scenarios while this benchmark is aimed to be a worst-case scenario. 
Real world applications are most likely achieve performances between those values.

Since both frameworks use different data structures and algorithms, significant adaptations to the code base are required for a project to migrate to NanoVDB. 
The expected increase in performance for CPU based systems may be considered too small to justify the amount of necessary work.
However for machines with a dedicated GPU a switch to NanoVDB can be very attractive especially for large simulations using multiple GPUs.
NanoVDB-Kernels can be used interchangeably on CPU and GPU, i.e. the target platform is determined at compile time.
Therefore very little adaptations to the codebase are required to launch kernels on either CPUs or GPUs.
This also allows to easily combine both platforms to perform calculations simultaneously thus effectively combining the performance of both platforms.

In its current form NanoVDB only supports static grids.
This means that the whole data structure must me re-computed if the level-set changes.
Depending on the complexity of the transformation it might be easier to convert the grid to the OpenVDB format before applying the transformation as OpenVDB is a more feature-rich and mature frameworks. 
For example generating platonic solids in OpenVDB and transforming them to NanoVDB is significantly faster than generating them natively because only OpenVDBs generation functions are multi-threaded. 
Depending on the complexity of the simulation this process may require significant amounts of time.
However within the context of semiconductor process simulation the majority of time is spent on calculating ray intersections as there is no need to finish all calculations within a given frame time. 

It should also be noted that a GPU's Video-RAM is usually much smaller than regular RAM. 
This not only limits the maximum grid size but also the number of rays that can be stored or buffered on GPU.
NanoVDB's Ray-class has a memory-footprint of 56 bytes per ray. 
The GPU used for this benchmark has 16GB of memory and can therefore theoretically hold up to 285 million rays (omitting grid, kernels, etc.).
Using the measured performance of 117 MRps the computation of all rays would complete within 2.5s.
In order to go beyond that limit new rays need to be generated during the simulation.
Performing this step on CPU would cause an additional load 6.4 GB/s on the PCI bus.
This number also increases if more than one GPU is used. 
Therefore ray generation should be performed on GPU as well.
However this is not covered by this benchmark as this problem was not considered in the design phase of the experiment. 

\subsection{Performance improvements}
Nvidia GPUs follow a SIMT \footnote{Single Instruction Multiple Thread} hardware architecture where threads are grouped in 'warps'.
Each warp usually contains 32 synchronized threads which all execute the same instruction.
In order to execute \texttt{if-else}-statements both branches have to be evaluated by a warp consecutively while a mask is used to disable the the false-path in each thread. \cite[Chapter~3.6.3]{mccool}
Therefore every branch is potentially wasteful and should be avoided by using branchless patterns.


Alg. \ref{cod:ZeroCrossingCore} shows the core loop if NanoVDB's raytracing function which contains 2 \texttt{if}-branches.
Additional branches are present in various subroutines such as \texttt{RoundDown()} or \texttt{hdda.update()}.
Furthermore two \texttt{while}-loops are present which in the context of warps behave just like branches.
I.e.: Each warp must 'wait' until all 32 threads have finished the loop.

\begin{lstlisting}[caption={core loop of NanoVDB's raytraciong function}]
while (hdda.step()) {
    ijk = RoundDown<Coord>(ray(hdda.time() + Delta));
    hdda.update(ray, acc.getDim(ijk, ray));
    if (hdda.dim() > 1 || !acc.isActive(ijk))
        continue; // either a tile value or an inactive voxel
    while (hdda.step() && acc.isActive(hdda.voxel())) { // in the narrow band
        v = acc.getValue(hdda.voxel());
        if (v * v0 < 0) { // zero crossing
            ijk = hdda.voxel();
            t = hdda.time();
            return true;
        }
    }
}
\end{lstlisting}
\label{cod:ZeroCrossingCore}



The outer loop is used while the ray is outside the narrow band. Once the ray enters a narrow band the inner loop is active and the value of each voxel is evaluated to detect an intersection.
Therefore as long as any ray of a warp remains within a narrow-band, the computation of all other rays is essentially halted.

Modern compilers are likely able to optimize simple conditions such as \texttt{RoundDown()} to prevent branching.
However more complex operations may increase the likelihood of wasteful branches. 
One way to achieve this could be by separating the nested while-loops such that only rays outside of the narrow band are processed. 
Once a ray enters the narrow-band it is handled by a different warp or kernel.

As mentioned above, NanoVDBs Ray-Class has a memory footprint of 56 bytes \footnote{Measured using sizeof()}.
A ray consists at minimum of a pair of 3D-vectors or a set of 6 floating point numbers. 
Using single precision floats (4 bytes) each ray requires at least 24 bytes.
This means that NanoVDB rays require almost twice as much memory.
Therefore a better data structure could be introduced to reduce the memory footprint. 
This new structure can also be used to make use of more efficient memory alignments.


Since 2018 NVIDIA is also selling GPUs with hardware accelerated raytracing capabilities (RTX 20X0 and 30X0 series).
Due to its proprietary license the exact functionality of this technology is not part of the public domain but the technology seems to be a hardware implementation of NVIDIA's OptiX pipeline.
A recent analysis \cite{sanzharov2020survey} also shows that a performance increase of up to one order of magnitude is possible in certain raytracing scenarios.
However additional research and testing is required to determine if NanoVDB is compatible with this technology.

\subsection{Accuracy}
As shown in Fig. \ref{fig:results_accuracy} NanoVDB is limited by the resolution of the grid when calculating ray intersections.
The output parameters of \texttt{nanovdb::ZeroCrossing} for coordinates and time are hardcoded to \texttt{int} and \texttt{float} (see Alg. \ref{cod:ZeroCrossing}).

Despite the \texttt{t} being returned as a floating point number the resolution is still limited by the voxel size and NanoVDB does not perform additional steps to improve the accuracy of the ray intersection. 
The result basically represents the number of voxels between the ray source and the intersection point.

\subsection{Accuracy improvements}

It should however be possible to increase the accuracy of Alg. \ref{cod:ZeroCrossingCore} with some modifications.
For example the value of neighboring voxels can be used to approximate the intersection point within a voxel.
This operation only needs to be performed once an intersection is found. Thus the performance impact should be minimal.

\begin{lstlisting}[caption={Footprint of NanoVDB's raytracing function. Types for coordinates (ijk) and time (t) are hardcoded. The function returns false if no intersection has been detected}]
    template<typename RayT, typename AccT>
    inline __hostdev__ bool ZeroCrossing(
        RayT& ray,  // IN: Ray used for intersection
        AccT& acc,  // IN: Accessor object to the grid
        Coord& ijk, // OUT: Intersection coordinates (int[3])
        typename AccT::ValueType& v, // OUT: grid value at intersection
        float& t    // OUT: Intersection time
    )
    \end{lstlisting}
    \label{cod:ZeroCrossing}







