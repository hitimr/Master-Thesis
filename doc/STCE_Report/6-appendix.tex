\section{Appendix}

\begin{lstlisting}[caption={NanoVDBs Coord class. Note that datatypes are hardcoded to integers}]
    // NanoVDB.h Line 859
    class Coord
    {
        int32_t mVec[3]; // private member data - three signed index coordinates
    public:
        using ValueType = int32_t;
        using IndexType = uint32_t;
        ...
\end{lstlisting}
\label{cod:nano_coords}

\begin{lstlisting}[caption={NanoVDBs raytracing implementation. }]  
template<typename RayT, typename AccT>
inline __hostdev__ bool ZeroCrossing(RayT& ray, AccT& acc, Coord& ijk, typename AccT::ValueType& v, float& t)
{
    if (!ray.clip(acc.root().bbox()) || ray.t1() > 1e20)
        return false; // clip ray to bbox
    static const float Delta = 1.0001f;
    ijk = RoundDown<Coord>(ray.start()); // first hit of bbox
    HDDA<RayT, Coord> hdda(ray, acc.getDim(ijk, ray));
    const auto        v0 = acc.getValue(ijk);
    while (hdda.step()) {
        ijk = RoundDown<Coord>(ray(hdda.time() + Delta));
        hdda.update(ray, acc.getDim(ijk, ray));
        if (hdda.dim() > 1 || !acc.isActive(ijk))
            continue; // either a tile value or an inactive voxel
        while (hdda.step() && acc.isActive(hdda.voxel())) { // in the narrow band
            v = acc.getValue(hdda.voxel());
            if (v * v0 < 0) { // zero crossing
                ijk = hdda.voxel();
                t = hdda.time();
                return true;
            }
        }
    }
    return false;
}
\end{lstlisting}
\label{cod:nano_zero_crossing}
